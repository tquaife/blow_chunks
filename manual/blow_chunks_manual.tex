\documentclass[11pt,a4paper,oldfontcommands]{memoir}
\usepackage[utf8]{inputenc}
\usepackage[T1]{fontenc}
\usepackage{microtype}
\usepackage[dvips]{graphicx}
\usepackage{xcolor}
\usepackage{times}
\usepackage{listings}

\newcommand{\blowchunks}{\texttt{blow\_chunks }}

\definecolor{dkgreen}{rgb}{0,0.6,0}
\definecolor{gray}{rgb}{0.5,0.5,0.5}
\definecolor{mauve}{rgb}{0.58,0,0.82}

\lstset{frame=tb,
%  language=bash,
  aboveskip=3mm,
  belowskip=3mm,
  showstringspaces=false,
  columns=flexible,
  basicstyle={\small\ttfamily},
  numbers=none,
  numberstyle=\tiny\color{gray},
  keywordstyle=\color{blue},
  commentstyle=\color{dkgreen},
  stringstyle=\color{mauve},
  breaklines=true,
  breakatwhitespace=true,
  tabsize=3
}




\usepackage[
breaklinks=true,colorlinks=true,
%linkcolor=blue,urlcolor=blue,citecolor=blue,% PDF VIEW
linkcolor=black,urlcolor=black,citecolor=black,% PRINT
bookmarks=true,bookmarksopenlevel=2]{hyperref}

\usepackage{geometry}
% PDF VIEW
% \geometry{total={210mm,297mm},
% left=25mm,right=25mm,%
% bindingoffset=0mm, top=25mm,bottom=25mm}
% PRINT
\geometry{total={210mm,297mm},
left=20mm,right=20mm,
bindingoffset=10mm, top=25mm,bottom=25mm}

\OnehalfSpacing
%\linespread{1.3}

%%% CHAPTER'S STYLE
\chapterstyle{bianchi}
%\chapterstyle{ger}
%\chapterstyle{madsen}
%\chapterstyle{ell}
%%% STYLE OF SECTIONS, SUBSECTIONS, AND SUBSUBSECTIONS
\setsecheadstyle{\Large\bfseries\sffamily\raggedright}
\setsubsecheadstyle{\large\bfseries\sffamily\raggedright}
\setsubsubsecheadstyle{\bfseries\sffamily\raggedright}


%%% STYLE OF PAGES NUMBERING
%\pagestyle{companion}\nouppercaseheads 
%\pagestyle{headings}
%\pagestyle{Ruled}
\pagestyle{plain}
\makepagestyle{plain}
\makeevenfoot{plain}{\thepage}{}{}
\makeoddfoot{plain}{}{}{\thepage}
\makeevenhead{plain}{}{}{}
\makeoddhead{plain}{}{}{}


\maxsecnumdepth{subsection} % chapters, sections, and subsections are numbered
\maxtocdepth{subsection} % chapters, sections, and subsections are in the Table of Contents


%%%---%%%---%%%---%%%---%%%---%%%---%%%---%%%---%%%---%%%---%%%---%%%---%%%

\begin{document}

%%%---%%%---%%%---%%%---%%%---%%%---%%%---%%%---%%%---%%%---%%%---%%%---%%%
%   TITLEPAGE
%
%   due to variety of titlepage schemes it is probably better to make titlepage manually
%
%%%---%%%---%%%---%%%---%%%---%%%---%%%---%%%---%%%---%%%---%%%---%%%---%%%
\thispagestyle{empty}

{%%%
\sffamily
\centering
\Large

~\vspace{\fill}

{\huge 
\texttt{blow\_chunks} 
}

{\LARGE
user manual
}


\vspace{2.5cm}

{\LARGE
Tristan Quaife
}

\vspace{3.5cm}


\vspace{\fill}

\today

%%%
}%%%

\cleardoublepage
%%%---%%%---%%%---%%%---%%%---%%%---%%%---%%%---%%%---%%%---%%%---%%%---%%%
%%%---%%%---%%%---%%%---%%%---%%%---%%%---%%%---%%%---%%%---%%%---%%%---%%%

\tableofcontents*

\clearpage

%%%---%%%---%%%---%%%---%%%---%%%---%%%---%%%---%%%---%%%---%%%---%%%---%%%
%%%---%%%---%%%---%%%---%%%---%%%---%%%---%%%---%%%---%%%---%%%---%%%---%%%

\chapter{Quick start}

\section{Getting the code}

You can download the code from \texttt{https://github.com/tquaife/blow\_chunks} or, better, use \texttt{git} to check out the repository:

\begin{lstlisting}
git checkout https://github.com/tquaife/blow_chunks
\end{lstlisting}

\section{Compilation}

\blowchunks has no external dependencies so, assuming you have a C compiler and \texttt{make} installed, it should be as easy as opening a terminal, changing directory to the location of the source code and typing:

\begin{lstlisting}
make
\end{lstlisting}

\noindent
If you don't have these tools, search for how to install C development tools for your system.

\section{A first sound}

\blowchunks reads ascii text from the standard input and writes wav files to the standard output. A very simple example of this is as follows. Open the terminal and type:

\begin{lstlisting}
echo sin 220 1 1|./blow_chunks > a.wav
\end{lstlisting}

\noindent
This command produces a one second sample of the note A at 220Hz using a sine wave oscillator. On most linux systems, you can listen to the sound by typing:

\begin{lstlisting}
open a.wav
\end{lstlisting}

\noindent
which will use your default media player to play the wav file. Alternatively, if you have \texttt{SoX} installed, you can use the \texttt{play} command:

\begin{lstlisting}
play a.wav
\end{lstlisting}



\noindent
You can also use \texttt{play} to directly preview the sound without writing to a file:

\begin{lstlisting}
echo sin 220 1 1|./blow_chunks|play -
\end{lstlisting}


\section{Basic \blowchunks instructions}

What's going on in the command described in the previous section? The command being passed to \blowchunks is ``\texttt{sin 220 1 1}'', 
which is an example of a oscillator, the main building block of sounds in \blowchunks. Most oscillator commands look something like:

\begin{lstlisting}
oscillator_name frequency phase volume
\end{lstlisting}

\noindent
The different oscillators are described later in this document, but valid examples include \texttt{sin}, \texttt{sqr} and \texttt{tri} which are a sine wave, a square wave and a triangular wave, respectively. They all take the same arguments. The \texttt{frequency} argument controls the pitch, \texttt{phase} controls the phase (but will not have a noticeable effect in these simple examples) and volume controls the volume of the sound. All of these arguments can take on any floating point value, including negative numbers, but typically \texttt{frequency} is a value above 0.0Hz and less than 22000.0Hz, and \texttt{phase} and \texttt{volume} will have values between 0.0 and 1.0.


\section{Modulating modulators}

A key concept in \blowchunks is that the arguments to oscillators (i.e. the numbers following the name of the oscillator) can be replaced with more oscillators, which we refer to as modulators. This is done by following the numeric argument with curly brackets and specifying a new oscillator inside the brackets. This allows \blowchunks to implement a variety of synthesis techniques, such as FM (or ``frequency modulation'') synthesis. For example, we can modify our original example as follows:

\begin{lstlisting}
echo sin 220 {sin 220 1 1} 1 1|./blow_chunks >afm220.wav
\end{lstlisting}

\noindent
Here, the frequency of our original oscillator is being modulated at its original frequency. Listen to the resulting sound. Basic FM synthesis is often described as making the original waveform sound ``brighter'' and was the foundation of many synths of the 1980's. Of course you don't need to modulate the frequency by itself, you can also try out other frequencies, potentially triggering all sorts of side-harmonics, some of which are less pleasing to the ear than others:

\begin{lstlisting}
echo sin 220 {sin 356 1 1} 1 1|./blow_chunks >afm356.wav
\end{lstlisting}

\noindent
It is possible to keep nesting modulators indefinitely, at least within the constraints of your computer's memory, which will be a \emph{lot} of modulators on modern computers. In the following example, we will use a low-frequency triangular-wave oscillator to modulate the amplitude of the FM effect. Not that there's nothing special about an oscillator that makes it ``low frequency''; it is just a standard \blowchunks oscillator with a pitch that is below our hearing range. In this example, we will use a frequency of 1Hz:

\begin{lstlisting}
echo sin 220 {sin 356 1 1{tri 1 1 1}} 1 1|./blow_chunks >afm356_fade.wav
\end{lstlisting}

\noindent
N.B. The above should all be typed on one line.

\noindent
Have a listen and convince yourself of how the above command is being used to generate dynamics in the sound. Also note how quickly we have been able to build up something that is interesting. At this stage, it is worth having a go at adding modulators into the above sound and experimenting with the different effects.

\section{Basic sound duration control}

At this point, you may want to make sounds be a bit longer than one second, so you can explore the effects of different low frequency modulators (LFOs). For basic sounds, this can be controlled using the optional argument \texttt{-d} to \blowchunks. This creates the previous sound for 2.5 seconds we can do the following:

\begin{lstlisting}
echo sin 220 {sin 356 1 1{tri 1 1 1}} 1 1|./blow_chunks -d 2.5 > afm356_fade_2p5s.wav
\end{lstlisting}


\section{Additive synthesis and working with files}

A well as providing a flexible modulation framework, \blowchunks allows multiple waveforms to be added together to make more complex sounds. However, it becomes cumbersome to do this using the command line, so it is better to write the instructions to a text file. Open a text editor and create a file called \texttt{cmajor.txt} with the following contents:

\begin{lstlisting}
;a C major chord
sn3 261.6 1 1
sn3 329.6 1 1
sn3 392.0 1 1
\end{lstlisting}

\noindent
We can now generate the corresponding wav file as follows:

\begin{lstlisting}
./blow_chunks -d 2.5 < cmajor.txt > cmajor.wav
\end{lstlisting}


\noindent
Note that \blowchunks supports comments. Anything after a ``;'' character until the end of the line is ignored. This is useful for leaving notes explaining what different parts of your file are doing. Blank lines are also ignored, which can be helpful for clearly separating different blocks of instructions. The \texttt{sn3} oscillator is described later, but produces a more satisfying result than \texttt{sin} in this example.

\section{Stereo}

Top level oscillators, i.e. those not enclosed in curly brackets ``\{\}'' can posses more than one volume value. By specifying two volume values the resulting sound file is in stereo (the resulting wav file will have two \emph{channels}). Note that all top level oscillators must have the same number of volume values or \blowchunks will throw an error and exit. Similarly, all oscillators being used as modulators can only have a single volume value. In principle \blowchunks supports an arbitrary number of channels, so it is possible to create quadraphonic sounds etc, but you will only hear these effects if you have an audio set up that can play them.

The simplest example if to pan a sound to one channel:

\begin{lstlisting}
;C panned hard left
sn3 261.6 1 1 0
\end{lstlisting}

By modulating the volume channels it's possible to generate effects in stereo. In the following example a sin wave, with two channels, has the volume of each channel modulated by a low frequency sine oscillator with a phase difference of $180^\circ$:

\begin{lstlisting}
;A sweeping from right to left
sin 220 1 1 {sin 1 1 1} 1 {sin 1 0.5 1}
\end{lstlisting}

It's also possible to use LFO modulated channels to create arpeggiators. A good choice of LFO for this pupose is the smoothed square wave modulator \texttt{sqx}. The following example is a stereo arpeggiated C major chord:

\begin{lstlisting}
;C major stereo arpeggiator
;suggest generating at least 6 second sample
sn3 261.6 1 1 {sqx 1 0 1} 1 {sqx 1 0.5 1}
sn3 329.6 1 1 {sqx 1 0.16 1} 1 {sqx 1 0.67 1}
sn3 392.0 1 1 {sqx 1 0.33 1} 1 {sqx 1 0.83 1}
\end{lstlisting}


\section{Variables}

\blowchunks supports variables, including user defined and predefined ones (which can be overwritten by the user). The strength of user defined variables is largely in conjunction with other parts of the \blowchunks interface, but the predefined variables have some more obvious applications. The frequency values of notes in the twelve tone equal temperament tuning with A referenced to 440Hz are all stored as variables. All variables begin with the ``\$'' character, and the frequency values can be retrived by typing, for example, \$A2 (which is 110Hz). Flats are specified with a ``b'' character (e.g. \$Ab2) and sharps with an ``s'' character (e.g. \$As2). That means we can rewrite input files so that the contain the note names, which is far more convenient. The C major chord example now becomes:

\begin{lstlisting}
;C major stereo arpeggiator, using in-built 
;variables to specify the notes
;suggest generating at least 6 second sample
sn3 $C4 1 1 {sqx 1 0 1} 1 {sqx 1 0.5 1}
sn3 $E4 1 1 {sqx 1 0.16 1} 1 {sqx 1 0.67 1}
sn3 $G4 1 1 {sqx 1 0.33 1} 1 {sqx 1 0.83 1}
\end{lstlisting}

To provide and example of the user defined variables, imagine you wanted to experiment with the speed of the arpeggiation in the previous example. It would be necessary to change the frequency of each of the LFO modulator in turn the the same thing. In more complex sounds this could become onerous and a likely source of error. Instead, we can set variable to hold that value, as follows: 
 
\begin{lstlisting}
;A slower C major stereo arpeggiator, using in-built 
;variables to specify the notes and a speed variable
;suggest generating at least 6 second sample
LfoSpeed=0.8
sn3 $C4 1 1 {sqx $LfoSpeed 0 1} 1 {sqx $LfoSpeed 0.5 1}
sn3 $E4 1 1 {sqx $LfoSpeed 0.16 1} 1 {sqx $LfoSpeed 0.67 1}
sn3 $G4 1 1 {sqx $LfoSpeed 0.33 1} 1 {sqx $LfoSpeed 0.83 1}
\end{lstlisting}

It's good practice to name variables is a meaningful way, but any combination of alphanumeric characters are permitted. No spaces or punctuation can be used, and the variables are case sensitive.


\section{The @sequence command}

Commands in \blowchunks start with the ``@'' character and probably the most import is the @sequence command which allows oscillators to start and stop at different times. Commands typically take a number of arguments and @sequence requires two: the start time and duration, in seconds, of the oscillators that follow it until the next @sequence command is encountered.

\begin{lstlisting}
;C major arpeggio
@sequence 0.0 0.2
sn3 $C4 1 1 1 
@sequence 0.2 0.2
sn3 $E4 1 1 1 
@sequence 0.4 0.2
sn3 $G4 1 1 1 
@sequence 0.6 0.2
sn3 $C5 1 1 1 
@sequence 0.8 0.2
sn3 $G4 1 1 1 
@sequence 1.0 0.2
sn3 $E4 1 1 1 
@sequence 1.2 0.4
sn3 $C4 1 1 1 
\end{lstlisting}

@sequence commands do not need to come in any particular order, so it is possible to organise the input file in whatever way is most intuitive. The duration of the wav file is set to whichever is longer between the maximum length of the sequences \emph{or} the duration specified on the command line. 

\end{document}
